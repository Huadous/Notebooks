\documentclass[11pt]{report}
\usepackage[default,regular,black]{sourceserifpro}
\usepackage{xcolor}
\usepackage{soul}
\usepackage[T1]{fontenc}
\usepackage[utf8]{inputenc}
\usepackage{graphicx}
\usepackage{amsmath}
\usepackage{tcolorbox}
\tcbuselibrary{theorems}
\tcbuselibrary{skins,breakable}
\usepackage{framed}
\usepackage[a4paper,margin=1in]{geometry}
\usepackage[raggedleft]{titlesec}
\usepackage{fancyhdr}
\usepackage{eco}
\usepackage[hidelinks]{hyperref}
\usepackage{bookmark}
\usepackage[ruled,linesnumbered]{algorithm2e}
\usepackage{fancybox}
\usepackage{calligra}
\usepackage{amsthm}
\usepackage{calc}
\usepackage{listings}
\usepackage{lstautogobble}  % Fix relative indenting
\usepackage{pifont}% http://ctan.org/pkg/pifont
\newcommand{\cmark}{\ding{51}}%
\newcommand{\xmark}{\ding{55}}%


\definecolor{stanfordred}{RGB}{140, 21, 21}
\definecolor{bluekeywords}{rgb}{0.13, 0.13, 1}
\definecolor{greencomments}{rgb}{0, 0.5, 0}
\definecolor{redstrings}{rgb}{0.9, 0, 0}
\definecolor{graynumbers}{rgb}{0.5, 0.5, 0.5}
\definecolor{codecell}{HTML}{f1f3f4}

\titleformat{\chapter}[display]{\raggedleft\huge\color{stanfordred}\bfseries}{CHAPTER \Huge\textcolor{black}{\oldstylenums{\thechapter}}}{3em}{}{}
\titleformat{\section}[block]{\raggedright\huge\color{stanfordred}}{\huge\color{black}{\oldstylenums{\thesection}}}{1em}{}
\titleformat{\subsection}[block]{\raggedright\normalfont\color{stanfordred}}{\thesubsection}{1em}{}
\titleformat*{\subsubsection}{\large}
\titleformat*{\paragraph}{\large}
\titleformat*{\subparagraph}{\large}

\fancypagestyle{plain}{
    \fancyhf{}
    \fancyfoot[R]{\sffamily\thepage}
    \renewcommand{\headrulewidth}{0pt}
}

\fancypagestyle{StanfordOnline}{
    \fancyhf{}
    \renewcommand\headrulewidth{.0pt}
    \renewcommand\footrulewidth{.0pt}
    \fancyfoot[L]{\leftmark}
    \fancyfoot[R]{\sffamily\thepage}
    \fancyhead[L]{\textcolor{stanfordred}{Stanford} | Online}
    \fancyhead[R]{Simon Hua}
    \setlength{\headheight}{18pt}
}
\pagestyle{StanfordOnline}

\lstset{
    autogobble,
    columns=fullflexible,
    showspaces=false,
    showtabs=false,
    breaklines=true,
    showstringspaces=false,
    breakatwhitespace=true,
    escapeinside={(*@}{@*)},
    commentstyle=\color{greencomments},
    keywordstyle=\color{bluekeywords},
    stringstyle=\color{redstrings},
    numberstyle=\color{graynumbers},
    basicstyle=\ttfamily\footnotesize,
    frame=l,
    framesep=12pt,
    xleftmargin=12pt,
    tabsize=4,
    captionpos=b
}
\renewcommand{\labelitemi}{\(\bullet\)}

\newenvironment{cell}{%
	% {\calligra Proof:}
	\tcolorbox[blanker,breakable,left=5mm,
	before skip=10pt,after skip=10pt,
	borderline west={1mm}{0pt}{stanfordred}]
}%
{\endtcolorbox}

\title{Compilers}
\author{Simon Hua}
\date{June 2021}

\newcommand{\courtitle}{Compilers}
\newcommand{\cournum}{\textcolor{stanfordred}{Stanford} | Online}
\newcommand{\prof}{Alex Aiken}

\newcommand*{\mytitle}{
\begin{titlepage} % Suppresses headers and footers on the title page

    \centering % Centre all text

    %------------------------------------------------
    %   Title and subtitle
    %------------------------------------------------

    \setlength{\unitlength}{0.6\textwidth} % Set the width of the curly brackets above and below the titles

    {\color{stanfordred}\resizebox*{\maxof{\widthof{\Huge\courtitle}}{\unitlength}}{\baselineskip}{\rotatebox{90}{$\}$}}}\\[\baselineskip] % Top curly bracket

    \textcolor{stanfordred}{\textit{\Huge \sffamily \courtitle}}\\[\baselineskip] % Title

    {\Large \sffamily \cournum}\\ % Subtitle or further description

    {\color{stanfordred}\resizebox*{\maxof{\widthof{\Huge\courtitle}}{\unitlength}}{\baselineskip}{\rotatebox{-90}{$\}$}}} % Bottom curly bracket

    \vfill % Whitespace between the title and the author name

    %------------------------------------------------
    %   Author
    %------------------------------------------------

    {\Large \sffamily  \prof}\\ % Author name

    \vfill % Whitespace between the author name and the publisher logo

    %------------------------------------------------
    %   Publisher
    %------------------------------------------------

    {\small \sffamily \LaTeX ed by
    {Simon Hua}} % Publisher name

\end{titlepage}}

% \usepackage{tikz}

% \newcommand\codecell[1]{\begin{tikzpicture}
%     \node [fill=codecell, rounded corners=5pt, anchor=south] {\footnotesize{#1}};
% \end{tikzpicture}}

% \lstdefinestyle{codecell}{
%     basicstyle=\footnotesize,
%     backgroundcolor=codecell,
%     breaklines=true,
%     tabsize=4,
%     upquote=true
% }
\newcommand{\codecell}[1] {\colorbox{codecell}{\parbox{\widthof{#1}}{#1}}}

% \def\checkmark{\tikz\fill[scale=0.4](0,.35) -- (.25,0) -- (1,.7) -- (.25,.15) -- cycle;} 

\lstdefinestyle{code}{
    basicstyle=\fontfamily{cmtt}\selectfont,
    upquote=true,
    tabsize=4,
    frame=none,
    mathescape=true
}
\lstnewenvironment{code}{\lstset{style=code}}{}
% \newcommand\codecell[1]{\colorbox{codecell}{\parbox[t]{\widthof{\footnotesize{#1}}}{\lstinline[style=codecell,mathescape]`#1`}}}

\begin{document}
    % \maketitle
    %     \centering
    %     {
    %         \normalfont
    %         \textcolor{stanfordred}{Stanford} | Online
    %     }
    %     by Alex Aiken

    %     \LaTeX{}
    %     \thispagestyle{empty}
    % \newpage
    % \clearpage
    % \pagenumbering{arabic}
    % % \begin{abstract}
    % %     This a note for Stanford Online course named Compiler. I'd like to share my note with you guys.
    % %     After three months of intership. I find the importance of Compiler when we meet any problem strange.
    % %     Thus, learning makes me feeling better.
    % % \end{abstract}
    % \tableofcontents
    \mytitle

    \hypersetup{linkbordercolor=white}
    \tableofcontents
    
    \newpage
    % \pagestyle{headings}

    \chapter{Introduction}
    \section{Introduction}
    \begin{itemize}
        \item Compilers
            \begin{center}
                \begin{tabular}{c c c c c}
                    & & off-line & & \\
                    & & & & Data \\
                    & & & & \(\downarrow\) \\
                    program & \(\longrightarrow\) & \(\Ovalbox{Compilers}\) & \(\longrightarrow\) & exec \\
                    & & & & \(\downarrow\) \\
                    & & & & Output \\
                \end{tabular}
            \end{center}
            \begin{cell}
                1954 IBM develops the 704 \\
                software > hardware \\
                "Speedcoding"
                \begin{itemize}
                    \item 10-20x slower
                    \item 300 bytes = 30\% memory
                \end{itemize}
            \end{cell}

        \item Interpreters
            \begin{center}
                \begin{tabular}{c c c c c}
                    & & on-line & & \\
                    program & \(\longrightarrow\) & & & \\
                    & & \(\Ovalbox{Interpreters}\) & \(\longrightarrow\) & Output \\
                    Data & \(\longrightarrow\) & & & \\
                \end{tabular}
            \end{center}
            \begin{cell}
                FORTRAN 1(Formulas Translated) \\
                1954-1957 \\
                1958 50\% program in FORTRAN 1
            \end{cell}
    \end{itemize}

    \section{\textcolor{stanfordred}{Structure of Compiler}}
    5 phases
    \begin{enumerate}
        \item \textcolor{stanfordred}{Lexical Analysis}: divides program text into "words" or "tokens".
        \item \textcolor{stanfordred}{Parsing}: diagramming sentences.
        \item \textcolor{stanfordred}{Semantic Analysis}: try to understand "meaning". (hard)\\
        Compilers perform limited senmantic analysis to catch inconsistencies.\\
        \(\rightarrow\) Programming Languages define strict rules to avoid such ambiguities.
        \item \textcolor{stanfordred}{Optimization}: Antomatically modify prgrams so that they
        
        \(\rightarrow\) Run faster

        \(\rightarrow\) Use less space

        \(\rightarrow\) Reduce power consumption...

        \item \textcolor{stanfordred}{Code Generation(Code Gen)}
        
        \(\rightarrow\) Produces assembly code.(usually)

        \(\rightarrow\) A translation int another language.(Analgous to human translation)

    \end{enumerate}
    \begin{flushleft}
        FORTRAN 1: \(\Ovalbox{\quad\quad L \quad\quad}\quad\Ovalbox{\quad\quad P \quad\quad}\quad\Ovalbox{S}\quad\Ovalbox{\quad\quad O \quad\quad}\quad\Ovalbox{\quad\quad CG \quad\quad}\)\\
        \hspace*{\fill} \\
        \quad MODERN: \(\Ovalbox{ L }\quad\Ovalbox{ P }\quad\Ovalbox{\quad\quad S\quad\quad}\quad\Ovalbox{\quad\quad\quad\quad\quad\quad O \quad\quad\quad\quad\quad\quad}\quad\Ovalbox{ CG }\)\\
    \end{flushleft}
    
    \section{\textcolor{stanfordred}{The Economy of Programming Languages}}
    \begin{flushleft}
        \textcolor{stanfordred}{Question}
    \end{flushleft}
    \begin{enumerate}
        \item Why are there so many Programming Languages?
        
        \textcolor{stanfordred}{Application domians have distinctive / conflicting needs}.

        \begin{flushleft}
            \begin{tabular}{ l l l l l }
                & & \(\rightarrow\) Good Float Points & & \\
                Scientific Computing & \quad\quad\quad & \(\rightarrow\) Good Arrays & \quad\quad\quad & FORTRAN \\
                & & \(\rightarrow\) Parallelism & & \\
                \\ \hspace*{\fill} \\
                & & \(\rightarrow\) Persistence & & \\
                Business Application & & \(\rightarrow\) Report Generation & & SQL \\
                & & \(\rightarrow\) Data Analysis & & \\
                \\ \hspace*{\fill} \\
                & & \(\rightarrow\) Control of Resources & & \\
                Scientific Computing & & & & C/C++ \\
                & & \(\rightarrow\) Real TimeConstraints & & \\
            \end{tabular}
        \end{flushleft}
        \item Why are there new programming languages?
        
        \textcolor{stanfordred}{Claim: \textbf{Programmer training} is the dominant cost for a Programming Languages}

        \begin{enumerate}
            \item widely-used Languages are slow to change.
            \item Easy to start a new language. \(\longrightarrow\) Productivity > Training Cost
            \item Languages adopted to fill a void.
        \end{enumerate}
        \begin{flushleft}
            New languages tend to looks like old languages because of the Claim\\
            \(\rightarrow\) Reducing programming training, like Java vs C++.
        \end{flushleft}
        \item What is a good programming languages?\\
        \textcolor{stanfordred}{There is no universally accepted metric for language design. }
    \end{enumerate}

    \newpage
    \chapter{The Cool Programming Language}
    \section{Cool Overview}
    \textcolor{stanfordred}{COOL} (Classroom Object Oriented Language) \\
    Designed to be implemented in a short time and small enough for a one term project. \\
    Cool \(\longrightarrow\) MIPS(spim) \(\longrightarrow\) Assembly Language
    \section{Cool Examples}
    \begin{enumerate}
        \item example 1
        \begin{lstlisting}
        class Main inherits IO {
            main() : Object {
                out_string("Hello, world!\n")
            };
        };
        \end{lstlisting}
        \item  exmaple 2
        \begin{lstlisting}
            class Main inherits IO {
                main(): Object {{
                    out_string("Enter an integer greater-than or equal-to 0: ");
              
                    let input: Int <- in_int() in
                        if input < 0 then
                            out_string("ERROR: Number must be greater-than or equal-to 0\n")
                        else {
                            out_string("The factorial of ").out_int(input);
                            out_string(" is ").out_int(factorial(input));
                            out_string("\n");
                        }
                        fi;
                }};
              
                factorial(num: Int): Int {
                  if num = 0 then 1 else num * factorial(num - 1) fi
                };
            };
        \end{lstlisting}
    \end{enumerate}    

    \newpage
    \chapter{Lexical Analysis}
    \section{Lexical Analysis}
    \textcolor{stanfordred}{Token class}

    Token classes correspond to sets of strings.

    \begin{itemize}
        \item \textcolor{stanfordred}{Identifier}: strings of letters or digits, starting with a letter.
        \item \textcolor{stanfordred}{Integer/Number}: a non-empty string of digits.
        \item \textcolor{stanfordred}{Keyword}: "else" or "if" or "begin" or \dots
        \item \textcolor{stanfordred}{Whitespace}: a non-empty sequence of blanks, newlines, and tabs.
        \item \textcolor{stanfordred}{Operater}: like "==", "<" or ">" in cpp.
        \item \textcolor{stanfordred}{single character token(punctuatin mark)}: "(", ")", ";", "=".
    \end{itemize}

    Classify program substrings  according to role(token class).
    \\ \vspace*{\fill} \\
    \textcolor{stanfordred}{Communicate tokens to the parser} 

    \begin{center}
        \begin{tabular}{c c c c c}
            string & \(\longrightarrow\) & \Ovalbox{LA} & \(\longrightarrow\) & \Ovalbox{P}\\
            \(\text{foo = 42}\) & & & \(\underbrace{\langle\text{class, string}\rangle}_{token}\) & \\     
        \end{tabular}
    \end{center}
    For exmaple:
    \begin{center}
        \(\langle\text{Id, "foo"}\rangle\)\quad\(\langle\text{Op, "="}\rangle\)\quad\(\langle\text{Int, "42"}\rangle\)
    \end{center}
    \begin{cell}
        An implementation must do two things
        \begin{enumerate}
            \item Recognize substrings corresponding to tokens
            
            \(\rightarrow\) The lexemes
            \item Identify the token class of each lexeme 
            
            \begin{center}
                \(\underbrace{\langle \text{token class, lexeme}\rangle}_{\text{token}}\)
            \end{center}
        \end{enumerate}
    \end{cell}
    \section{Lexical Analysis Examples}
    \begin{enumerate}
        \item \textcolor{stanfordred}{FORTRAN rule: white space is insignificant.}
        \begin{enumerate}
            \item \codecell{VAR1} is the same as \codecell{VA R1}.
            \item \codecell{DO 5 I = 1,25}
            
            This is a loop and \codecell{I} is a variable from 1 t 25. The number 5 means execute the following 5 lines of the codes in the loop.

            \item \codecell{DO 5 I = 1.25}
            
            It's exactly the same as \codecell{DO5I = 1.25}.
        \end{enumerate}
        \begin{cell}
            \begin{enumerate}
                \item The goal is to partion the string. This is implemented by reading left-to-right, recognizing ne token at a time.
                \item "Lookahead" may be required to decided where one token ends and the next tken begins. 
            \end{enumerate}
        \end{cell}
        \item \textcolor{stanfordred}{PL/1(Programming langugae 1) keywords are not reserved }
        \begin{enumerate}
            % \item \codecell{\textcolor{stanfordred}{IF} ELSE \textcolor{stanfordred}{THEN} THEN = ELSE; \textcolor{stanfordred}{ELSE} ELSE = THEN;}
            \item \codecell{DECLARE(ARG 1, ..., ARG N)}
            
            Is \codecell{DECLARE} is a keyword or an array reference? Unbounded lookahead.
        \end{enumerate}

        \item \begin{itemize}
            \item C++ template syntax \\
                \codecell{Foo<Bar>}

            \item C++ template syntax \\
                \codecell{cin >{}> var;}
        \end{itemize}
        \codecell{Foo<Bar<Bazz>{}>} \\ 
        The Problem is how the compiler will deal with the code \codecell{>{}>}.
    \end{enumerate}
    
\begin{cell}
    \textcolor{stanfordred}{Summary} 

    \begin{itemize}
        \item The goal of lexical analysis is to
        \begin{enumerate}
            \item Partition the input string into lexemes.
            \item Identify the token of each lexeme.
        \end{enumerate}
        \item Left-to-right scan \(\Rightarrow\) lookahead sometimes required.
    \end{itemize}
\end{cell}

\section{Regular Languages}
The \textcolor{stanfordred}{Lexical structure} equals to \textcolor{stanfordred}{token class}.
We must say what set of strings is in a token class\\
\(\rightarrow\) Use regular languages. Then, we need to define the regular expression.

\begin{cell}
    \textcolor{stanfordred}{Define.} Regular Expression
    \begin{itemize}
        \item Single Character
        \begin{code}
            'c' = {"c"}
        \end{code}
        \item Epsilon
        \begin{code}
            $\epsilon$ = {""} $\neq$ $\phi$
        \end{code}
        \item Union
        \begin{code}
            A + B = {a|a $\in$ A} $\cup$ {b|b $\in$ B}
        \end{code}
        \item Concatenation
        \begin{code}
            AB = {ab| a $\in$ A $\wedge$ b $\in$ B}
        \end{code}
        \item Iteration
        \begin{code}
            A$^*$ = $\displaystyle\bigcup_{i\geq0}$ A$^i$
        \end{code} 
        \begin{code}
            e.g. A$^0$ = $\epsilon$, A$^i$ = $\underbrace{\text{A}\dots \text{A}}_{\text{i times}}$
        \end{code}
    \end{itemize}
\end{cell}
E.g. \quad $\Sigma$ = {0, 1}

\begin{enumerate}
    \item $\underbrace{\text{1}^*}_{\text{=1}^* + \text{1}}$ = $\displaystyle\bigcup_{i\geq0}$ 1$^i$ = "" + 1 + 11 + $\underbrace{\text{11} \dots \text{11}}_{\text{i times}}$ = all strings of 1's
    \item $\underbrace{\text{(1 + 0)}1}_{\text{=11 + 01}}$ = {ab|a $\in$ 1+0 $\wedge$ b $\in$ 1} = {11, 01}
    \item 0$^*$ + 1$^*$ = {0$^i$|$i\geq0$} $\cup$ {1$^i$|$i\geq0$}
    \item (0 + 1)$^*$ = $\displaystyle\bigcup_{i\geq0}$ (0 + 1)$^i$ = "", (0 + 1), (0 + 1)(0 + 1), $\underbrace{\text{(0 + 1)} \dots \text{(0 + 1)}}_{\text{i times}}$ \newline = all strings of 0's and 1's\newline = $\Sigma^*$
\end{enumerate}
\textcolor{stanfordred}{Question}\vspace{\baselineskip}

(0 + 1)$^*$1(0 + 1)$^*$, $\Sigma$ = {0, 1} $\Rightarrow$ (0 + 1)$\dots$(0 + 1)1(0 + 1)$\dots$(0 + 1)\vspace{\baselineskip}

$\rightarrow$ (01 + 11)$^*$(0 + 1)$^*\quad$\xmark

No guarantee of the first "(0 + 1)$^*$", like "10" is not allowed by "(01 + 11)$^*$".\vspace{\baselineskip}

$\rightarrow$ (0 + 1)$^*$(10 + 11 + 1)(0 + 1)$^*\quad$\cmark 

(10 + 11 + 1) guarantees the "1", $\Rightarrow$ 10 + 11 $\Leftrightarrow$ "1" + "0" or "1" + "1" = 1(1 + 0)$^*$.\vspace{\baselineskip}

$\rightarrow$ (1 + 0)$^*$1(1 + 0)$^*\quad$\cmark 

It's the same as (0 + 1)$^*$1(0 + 1)$^*$.\vspace{\baselineskip}

$\rightarrow$ (0 + 1)$^*(0 + 1)$(0 + 1)$^*\quad$\xmark

It's the same as (0 + 1)$^*$, and no guarantee the middle "1".\vspace{\baselineskip}

\begin{cell}
    \textcolor{stanfordred}{Summary}
    \begin{enumerate}
        \item \textcolor{stanfordred}{Regular expression}(syntax) specify \textcolor{stanfordred}{regular languages}(set of things)
        \item Five constructs
        \begin{enumerate}
            \item Two base cases
                empty and 1-character strings
            \item Three compound expressions
                union, concatenation, iteration
        \end{enumerate}
    \end{enumerate}
\end{cell}
 
\end{document}
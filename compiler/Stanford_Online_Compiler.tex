% \RequirePackage{filecontents}
% \begin{filecontents*}{\jobname.mst}
% 	delim_0 "\\IndexDotfill "
% 	delim_1 "\\IndexDotfill "
% 	headings_flag 1
% 	heading_prefix "  \\IndexHeading{"
% 		heading_suffix "}\n"
% \end{filecontents*}

\documentclass[12pt,titlepage]{report}
\usepackage[default,regular,black]{sourceserifpro}
\usepackage{xcolor}
\definecolor{StanfordRed}{RGB}{140, 21, 21}
\usepackage[T1]{fontenc}
\usepackage[utf8]{inputenc}
\usepackage{graphicx}
\usepackage{amsmath}
\usepackage{tcolorbox}
\usepackage{framed}
\usepackage[a4paper,width=150mm,top=25mm,bottom=25mm]{geometry}
\usepackage[raggedleft]{titlesec}
\usepackage{fancyhdr}
\usepackage{eco}
\usepackage[hidelinks]{hyperref}
\usepackage{bookmark}
\usepackage[ruled,linesnumbered]{algorithm2e}

\renewcommand{\thesection}{{\oldstylenums{\arabic{section}}}}
\renewcommand{\thesubsection}{{\oldstylenums{\arabic{section}.\arabic{subsection}}}}

\titleformat{\chapter}[block]{\raggedright\LARGE}{\thesection}{1em}{}
\titleformat{\section}[block]{\raggedright\LARGE}{\thesection}{1em}{}
\titleformat{\subsection}[block]{\raggedright\large}{\thesubsection}{1em}{}
\titleformat*{\subsubsection}{\large}
\titleformat*{\paragraph}{\large}
\titleformat*{\subparagraph}{\large}

\fancypagestyle{appendixFancy}{
    \fancyhf{}
    \renewcommand\headrulewidth{.1pt}
    \renewcommand\footrulewidth{.1pt}
    \fancyhead[L]{\leftmark}
    \fancyhead[C]{Simon Hua}
    \fancyhead[R]{\thepage}
    \fancyfoot[C]{\textcolor{StanfordRed}{Stanford} | Online by Alex Aiken}
}

\pagestyle{appendixFancy}

\makeatletter
\renewcommand{\boxed}[1]{\text{\fboxsep=.2em\fbox{\m@th$\displaystyle#1$}}}
\makeatother

\definecolor{lightgray}{rgb}{0.75,0.75,0.75}
 
\newenvironment{lightgrayleftbar}{%
  \def\FrameCommand{\textcolor{lightgray}{\vrule width 3pt} \hspace{3pt}}%
  \MakeFramed {\advance\hsize-\width \FrameRestore}}%
{\endMakeFramed}

\title{Compilers}
\author{Simon Hua}
\date{June 2021}

\begin{document}
    % \maketitle
    %     \centering
    %     {
    %         \normalfont
    %         \textcolor{StanfordRed}{Stanford} | Online
    %     }
    %     by Alex Aiken

    %     \LaTeX{}
    %     \thispagestyle{empty}
    % \newpage
    % \clearpage
    % \pagenumbering{arabic}
    % % \begin{abstract}
    % %     This a note for Stanford Online course named Compiler. I'd like to share my note with you guys.
    % %     After three months of intership. I find the importance of Compiler when we meet any problem strange.
    % %     Thus, learning makes me feeling better.
    % % \end{abstract}
    % \tableofcontents
    \maketitle
    {
        \hypersetup{linkbordercolor=white}
        \tableofcontents
    }
    \newpage
    \chapter{\textcolor{StanfordRed}{Introduction}}
    \section{\textcolor{StanfordRed}{Introduction}}
    \begin{itemize}
        \item Compilers
            \begin{center}
                \begin{tabular}{c c c c c}
                    & & off-line & & \\
                    & & & & Data \\
                    & & & & \(\downarrow\) \\
                    program & \(\longrightarrow\) & \(\boxed{C}\) & \(\longrightarrow\) & exec \\
                    & & & & \(\downarrow\) \\
                    & & & & Output \\
                \end{tabular}
            \end{center}
            \begin{lightgrayleftbar}
                1954 IBM develops the 704 \\
                software > hardware \\
                "Speedcoding"
                \begin{itemize}
                    \item 10-20x slower
                    \item 300 bytes = 30\% memory
                \end{itemize}
            \end{lightgrayleftbar}

        \item Interpreters
            \begin{center}
                \begin{tabular}{c c c c c}
                    & & on-line & & \\
                    program & \(\longrightarrow\) & & & \\
                    & & \(\boxed{I}\) & \(\longrightarrow\) & Output \\
                    Data & \(\longrightarrow\) & & & \\
                \end{tabular}
            \end{center}
            \begin{lightgrayleftbar}\noindent
                FORTRAN 1(Formulas Translated) \\
                1954-1957 \\
                1958 50\% program in FORTRAN 1
                
            \end{lightgrayleftbar}
    \end{itemize}

    \section{\textcolor{StanfordRed}{Structure of Compiler}}
    5 phases
    \begin{enumerate}
        \item \textcolor{StanfordRed}{Lexical Analysis}: divides program text into "words" or "tokens".
        \item \textcolor{StanfordRed}{Parsing}: diagramming sentences.
        \item \textcolor{StanfordRed}{Semantic Analysis}: try to understand "meaning". (hard)\\
        Compilers perform limited senmantic analysis to catch inconsistencies.\\
        \(\rightarrow\) Programming Languages define strict rules to avoid such ambiguities.
        \item \textcolor{StanfordRed}{Optimization}: Antomatically modify prgrams so that they
        
        \(\rightarrow\) Run faster

        \(\rightarrow\) Use less space

        \(\rightarrow\) Reduce power consumption...

        \item \textcolor{StanfordRed}{Code Generation(Code Gen)}
        
        \(\rightarrow\) Produces assembly code.(usually)

        \(\rightarrow\) A translation int another language.(Analgous to human translation)

    \end{enumerate}
    \begin{flushleft}
        FORTRAN 1: \(\boxed{\quad\quad L \quad\quad}\quad\boxed{\quad\quad P \quad\quad}\quad\boxed{S}\quad\boxed{\quad\quad O \quad\quad}\quad\boxed{\quad\quad CG \quad\quad}\)\\
        \hspace*{\fill} \\
        \quad MODERN: \(\boxed{ L }\quad\boxed{ P }\quad\boxed{\quad\quad S\quad\quad}\quad\boxed{\quad\quad\quad\quad\quad\quad O \quad\quad\quad\quad\quad\quad}\quad\boxed{ CG }\)\\
    \end{flushleft}
    
    \section{\textcolor{StanfordRed}{The Economy of Programming Languages}}
    \begin{flushleft}
        \textcolor{StanfordRed}{Question}
    \end{flushleft}
    \begin{enumerate}
        \item Why are there so many Programming Languages?
        
        \textcolor{StanfordRed}{Application domians have distinctive / conflicting needs}.

        \begin{flushleft}
            \begin{tabular}{ l l l l l }
                & & \(\rightarrow\) Good Float Points & & \\
                Scientific Computing & \quad\quad\quad & \(\rightarrow\) Good Arrays & \quad\quad\quad & FORTRAN \\
                & & \(\rightarrow\) Parallelism & & \\
                \\ \hspace*{\fill} \\
                & & \(\rightarrow\) Persistence & & \\
                Business Application & & \(\rightarrow\) Report Generation & & SQL \\
                & & \(\rightarrow\) Data Analysis & & \\
                \\ \hspace*{\fill} \\
                & & \(\rightarrow\) Control of Resources & & \\
                Scientific Computing & & & & C/C++ \\
                & & \(\rightarrow\) Real TimeConstraints & & \\
            \end{tabular}
        \end{flushleft}
        \item Why are there new programming languages?
        
        \textcolor{StanfordRed}{Claim: \textbf{Programmer training} is the dominant cost for a Programming Languages}

        \begin{enumerate}
            \item widely-used Languages are slow to change.
            \item Easy to start a new language. \(\longrightarrow\) Productivity > Training Cost
            \item Languages adopted to fill a void.
        \end{enumerate}
        \begin{flushleft}
            New languages tend to looks like old languages because of the Claim\\
            \(\rightarrow\) Reducing programming training, like Java vs C++.
        \end{flushleft}
        \item What is a good programming languages?\\
        \textcolor{StanfordRed}{There is no universally accepted metric for language design. }
    \end{enumerate}

    \newpage
    \chapter{}{\textcolor{StanfordRed}{The Cool Programming Language}}
    \section{\textcolor{StanfordRed}{Cool Overview}}
    
\end{document}inputenc